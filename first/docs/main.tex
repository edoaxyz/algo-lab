\documentclass[titlepage]{article}
\usepackage{graphicx}
\usepackage{titlesec}
\usepackage{algorithm}
\usepackage[noend]{algorithmic}
\usepackage{amsmath}
\usepackage{biblatex}
\usepackage{graphicx}
\usepackage{enumitem}
\usepackage{caption}
\usepackage{adjustbox}
\usepackage[a4paper,left=3.5cm,right=3.5cm,top=2cm,bottom=2cm]{geometry}
\usepackage[italian]{babel}

\newcommand{\hruleafter}[1]{#1\hrule}
\titleformat{\section}{\large\bfseries}{\thesection}{1em}{\hruleafter}
\algsetup{linenodelimiter=.}
\setlength\parindent{0pt}
\addbibresource{main.bib}
\graphicspath{ {./assets/} }
\makeatletter\renewcommand{\ALG@name}{Algoritmo}


\title{{\small\centerline {Relazione di}} Laboratorio di Algoritmi e Strutture Dati}
\author{Edoardo Grassi}
\date{Settembre 2023}


\begin{document}

\maketitle

\tableofcontents

\section*{Premessa}
Questa relazione fa parte di un progetto per il corso di Laboratorio
di Algoritmi e Strutture Dati ed è composto da due esercizi:
\begin{enumerate}
    \item Confronto tra diverse strutture dati per insiemi disgiunti nell'ambito della ricerca di componenti connesse in grafi non diretti
    \item Confronto tra \textit{insertion sort} e \textit{merge sort}
\end{enumerate}

In questa relazione viene analizzato e discusso solo il primo esercizio, 
la cui implementazione è stata scritta in codice \textit{Python}.

\section{Introduzione al problema}

Le strutture dati per insiemi disgiunti vengono utilizzate in algoritmi
nei quali sia necessario raggruppare diversi elementi sotto un unico rappresentante;
definito $\mathcal{S}$ l'insieme contenente tutti gli insiemi disgiunti,
possiamo descriverne tre operazioni fondamentali:


\begin{itemize}
    \item \textsc{Make-Set($x$)}: aggiunge a $\mathcal{S}$ un nuovo insieme
          $\mathcal{S}_i = \{x\}$ ($x$ non deve esistere all'interno di alcun altro insieme,
          altrimenti non sarebbero disgiunti)
    \item \textsc{Find-Set($x$)}: individua l'elemento rappresentante dell'insieme nel quale $x$ è contenuto
    \item \textsc{Union($x$, $y$)}: dati $x \in \mathcal{S}_x$ e $y \in \mathcal{S}_y$  
\end{itemize}
\section{Documentazione del codice}
Il codice per testare e confrontare le varie implementazioni della struttura dati è
contenuta nel modulo \texttt{disjoint\_sets}
\section{Risultati dei test}

I test sono stati eseguiti su un emulatore di terminale con sistema operativo basato sulla distribuzione \textit{Linux} \textit{ArchLinux},
attraverso un \textit{Python virtual environment} di versione \texttt{3.11.5}. L'hardware utilizzato è il seguente:
\begin{itemize}
    \item \textbf{CPU}: Intel(R) Core(TM) i3-5005U @ 2.00GHz
    \item \textbf{RAM}: 8GiB @ 1600Mhz
    \item \textbf{SSD}: KINGSTON SA400S3 128GB
\end{itemize}

Lanciando il seguente comando \texttt{time python .} nella directory in cui si trova il file \texttt{\_\_main\_\_.py} si ottiene,
oltre all'output generato dal codice \textit{Python} stesso anche il risultato del comando \texttt{time}:
\begin{center}
    \texttt{python .  40.59s user 3.99s system 39\% cpu 1:52.55 total}
\end{center}
i dati più significativi sono il \texttt{3.99s system} che indica il tempo di CPU usato dal programma e il \texttt{39\% cpu} che indica
il picco massimo di utilizzo della CPU generato per l'esecuzione del codice.\newline

Passando ai dati invece otteniamo tre grafici che si differenziano per i valori di copertura dei grafi:

\begin{figure}
    \centering
    \captionsetup{justification=centering}
    \includegraphics[width=\textwidth]{tests_results/cop0.2.png}
    \caption{Grafico dei tempi di esecuzione per la ricerca di componenti connesse in grafi non diretti con copertura 0.2\%}
\end{figure}


\begin{figure}
    \centering
    \captionsetup{justification=centering}
    \includegraphics[width=\textwidth]{tests_results/cop50.png}
    \caption{Grafico dei tempi di esecuzione per la ricerca di componenti connesse in grafi non diretti con copertura 50\%}
\end{figure}


\begin{figure}
    \centering
    \captionsetup{justification=centering}
    \includegraphics[width=\textwidth]{tests_results/cop100.png}
    \caption{Grafico dei tempi di esecuzione per la ricerca di componenti connesse in grafi non diretti con copertura 100\%}
\end{figure}



\end{document}
