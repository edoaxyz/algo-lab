\section{Introduzione al problema}

Le strutture dati per insiemi disgiunti vengono utilizzate in algoritmi
dove sia necessario raggruppare diversi elementi sotto un unico rappresentante;
definito $\mathcal{S}$ l'insieme contenente tutti gli insiemi disgiunti,
possiamo definire tre operazioni fondamentali su di esso:

\begin{itemize}
    \item \textsc{Make-Set($x$)}: aggiunge a $\mathcal{S}$ un nuovo insieme
          $\mathcal{S}_i = \{x\}$ ($x$ non deve esistere all'interno di alcun altro insieme,
          altrimenti non sarebbero disgiunti)
    \item \textsc{Find-Set($x$)}: individua l'elemento rappresentante dell'insieme nel quale $x$ è contenuto
    \item \textsc{Union($x$, $y$)}: dati $x \in \mathcal{S}_x$ e $y \in \mathcal{S}_y$
          ($x \ne y$ per quanto detto prima) $\mathcal{S} = (\mathcal{S} - \mathcal{S}_x - \mathcal{S}_y) \cup \mathcal{S}_\cup$
          dove $\mathcal{S}_\cup = \mathcal{S}_x \cup \mathcal{S}_y$ il cui rappresentante è
          dato da uno degli elementi in $\mathcal{S}_x$ o $\mathcal{S}_y$
\end{itemize}

\subsection{Ricerca di componenti connesse in grafi}

Un possibile utilizzo di queste strutture dati è nella ricerca di componenti
connesse in un \textbf{grafo non diretto} (o non orientato), ovvero quando le relative
connessioni non hanno direzione e pertanto collegano due vertici bidirezionalmente,
e questo verrà utilizzato per confrontare diverse implementazioni della
struttura dati.\newline

Dato $G(V,E)$ un grafo non diretto, con $V$ insieme dei vertici del grafo e $E$
l'insieme delle connessioni, l'algoritmo è il seguente:

\begin{algorithm}
    \caption{Ricerca di componenti connesse}\label{connectedComponentsAlg}
    \begin{algorithmic}[1]
        \FOR{$v \in G.V$} \label{alg:firstfor}
        \STATE \textsc{Make-Set($v$)}
        \ENDFOR
        \FOR{$(u,v) \in G.E$} \label{alg:secondfor}
        \IF{\textsc{Find-Set($u$)} $\ne$ \textsc{Find-Set($v$)}}
        \STATE \textsc{Union($u$, $v$)}
        \ENDIF
        \ENDFOR
    \end{algorithmic}
\end{algorithm}

\subsubsection{Copertura dei grafi}

Nei successivi capitoli useremo il termine \textit{copertura} per indicare una percentuale $c \in [0; 1]$ di connessioni rispetto
ad un numero massimo, definito in base al tipo copertura:

\begin{itemize}
    \item \textit{Copertura lineare}: è la percentuale di connessioni rispetto al numero di vertici del grafo
          e ci permetterà di stabilire che il numero di connessioni considerato è un $\Theta(v)$ con $v = |V|$;
          \begin{equation}
              e = |E| = c \cdot v = \Theta(v) \label{linearCoverage}
          \end{equation}
    \item \textit{Copertura quadratica}: è la percentuale di connessioni rispetto al numero massimo di connessioni
          possibili; essendo il grafo $G(V, E)$ non diretto si ha che
          \begin{equation} \label{maxedges}
              \max(|E|) = \frac{v(v-1)}{2} = \Theta(v^2) \text{ con } v = |V|
          \end{equation}
          quindi il numero di connessioni data una copertura quadratica $c$ è dato da:
          \begin{equation} \label{quadraticCoverage}
              e = |E| = c \cdot \frac{v(v-1)}{2} = \Theta(v^2)
          \end{equation}
          e per indicare il numero di connessioni nei costi computazionali utilizzeremo $\Theta(v^2)$.
\end{itemize}

\subsection{Implementazioni della struttura dati}
Come riportato precedentemente, esistono varie implementazioni delle strutture
dati per insiemi disgiunti che permettono di avere variazioni sui costi di complessità
computazionale.

\subsubsection{Liste concatenate}
Questa implementazione consiste nel rappresentare gli insiemi disgiunti tramite liste
concatenate. Ogni lista ha due puntatori, uno all'elemento iniziale della lista e uno
all'elemento finale, permettendo di ottenere in tempo costante il rappresentante di un
insieme, e di aggiungere elementi ad esso sempre in tempo costante. Ogni elemento della
lista contiene, oltre ai dati associati, anche un puntatore al suo rappresentante, così da
ottenere in tempo costante il rappresentante di qualsiasi elemento, e un puntatore al
successivo elemento della lista, come in qualsiasi lista concatenata.\newline

I costi computazionali delle operazioni base nei casi peggiori sono i seguenti:
\begin{itemize}
    \item \textsc{Make-Set($x$)} $\rightarrow O(1)$: dato che il costo di creazione di una nuova lista vuota è costante,
          e per quanto detto nel paragrafo precedente anche l'inserimento ha costo costante;
    \item \textsc{Find-Set($x$)} $\rightarrow O(1)$: per via del puntatore al rappresentante già presente all'interno
          di qualsiasi elemento;
    \item \textsc{Union($x$, $y$)} $\rightarrow O(n)$ con $n$ numero di elementi totali nella struttura dati: infatti,
          supponendo di concatenare tutti gli elementi della lista contenente $y$ alla lista contenente $x$, abbiamo il caso
          peggiore quando la lista di $x$ contiene solo $x$ stesso, e la lista di $y$ contiene i rimanenti $n-1$ elementi;
          dato che l'unione prevede di cambiare il puntatore al rappresentante per ogni elemento inserito nella lista finale,
          questa operazione ha tempo lineare, giustificando il risultato.
\end{itemize}

Il costo computazionale nel caso peggiore tiene conto del fatto che per $n$ elementi inseriti
all'interno della struttura dati potranno essere eseguite massimo $n-1$ \textsc{Union} il cui
numero di operazioni interne tenderà man mano verso $n-1$, quindi si considera il costo pari a:

\begin{equation}
    \Theta(n^2) \footnote{\fullcite{linkedList}} \label{LLCost}
\end{equation}

Considerando il problema della ricerca di componenti connesse su un grafo non diretto $G(V, E)$,
questo è composto da un ciclo \textit{for} alla riga \ref{alg:firstfor} che esegue $|V|$
operazioni di \textsc{Make-Set}, quindi crea $|V|$ elementi. Considerando \eqref{LLCost},
il costo totale è dato da:

\begin{equation}
    \Theta(v^2) \label{WorstLLCost}
\end{equation}

con $v = |V|$.

\subsubsection{Liste concatenate con euristica dell'unione pesata}
Questa implementazione riprende quella delle normali liste concatenate e cerca
di ridurre il costo dell'operazione di \textsc{Union} conservando nelle liste
un parametro che indica il numero di elementi contenuti in esse e questo viene
usato per unire la lista più corta a quella più lunga, riducendo il numero
di operazioni. Inoltre, la gestione di un contatore di elementi all'interno di
ogni lista ha costo costante, quindi i costi per le rimanenti operazioni
continuano ad essere i medesimi per le liste concatenate.\newline

In generale, data una sequenza di $m$ operazioni di \textsc{Make-Set}, \textsc{Find-Set}
and \textsc{Union}, $n$ delle quali sono operazioni di \textsc{Make-Set} (i.e., $n$
è il numero di elementi totali all'interno della struttura dati), il costo computazionale
è dato da:

\begin{equation}
    O(m+n\log_2n) \footnote{\fullcite{weightedUnionHeuristic}} \label{WUHCost}
\end{equation}

Analizzando il costo della struttura dati nell'algoritmo di ricerca di componenti connessi
su $G(V,E)$ grafo non diretto, si osserva che: $n$, il numero di \textsc{Make-Set} è dato
dal numero di iterazioni del ciclo \textit{for} alla riga \ref{alg:firstfor}; $m$ è invece
dato dalla somma di $n$ con il numero di \textsc{Find-Set} e \textsc{Union} del ciclo
\textit{for} alla riga \ref{alg:secondfor}, moltiplicate per il numero di iterazioni del
ciclo stesso:

\begin{equation}
    \label{WUHValues}
    \begin{aligned}
        n & = |V| = \Theta(|V|)= \Theta(v)                                          \\
        m & = n + |E| \cdot (2 + 1) = |V| + 3 |E| = \Theta(|V| + |E|) = \Theta(v+e)
    \end{aligned}
\end{equation}

con $v = |V|$ e $e = |E|$. Quindi per \eqref{WUHCost} il costo complessivo è dato da:

\begin{equation}
    \label{WUH2Cost}
    \begin{aligned}
        O(m+n\log_2n) & = O((v + 3 e) + v\log_2v) \\
                      & = O(v + 3 e + v\log_2v)   \\
                      & = O(3e+ v\log_2v)         \\
                      & = O(e + v\log_2v)
    \end{aligned}
\end{equation}

Considerando il costo rispetto al solo numero dei vertici usando la copertura lineare, per \eqref{linearCoverage} si ha:

\begin{equation}
    \label{linearWUH}
    \begin{aligned}
        O(e+v\log_2v) & = O(v+ v\log_2v) = O(v\log_2v)
    \end{aligned}
\end{equation}

mentre usando la copertura quadratica, per \eqref{quadraticCoverage} si ha:

\begin{equation}
    \label{quadraticWUH}
    \begin{aligned}
        O(e+v\log_2v) & = O(v^2+ v\log_2v) = O(v^2)
    \end{aligned}
\end{equation}

\subsubsection{Foreste di insiemi disgiunti con compressione dei cammini}
Questa implementazione prevede di rappresentare ogni elemento di ogni insieme disgiunto
come nodo di un albero, quindi è composto dal dato che contiene e da un puntatore al nodo
padre. Ogni insieme disgiunto è quindi rappresentato da un albero la cui radice diventa
automaticamente il rappresentante dell'insieme. Nella versione base, ovvero senza compressione
dei cammini, i costi delle operazioni base sono i seguenti:

\begin{itemize}
    \item \textsc{Make-Set($x$)} $\rightarrow O(1)$: prevede di creare un nuovo nodo senza padre;
    \item \textsc{Find-Set($x$)} $\rightarrow O(h_x)$ con $h_x$ altezza dell'albero in cui è contenuto $x$:
          questo perchè per individuare il rappresentante del nodo è necessario risalire tutto l'albero
          fino a giungere alla radice;
    \item \textsc{Union($x$, $y$)} $\rightarrow O(h_y)$ supponendo di porre $x$ come padre del rappresentante
          di $y$: infatti, per unire due alberi (o i due insiemi) basta porre come padre di uno dei due rappresentanti
          l'altro elemento, e automaticamente il rappresentante degli elementi dell'albero unito diviene il
          rappresentante dell'albero espanso.
\end{itemize}

La compressione del cammino cerca di attenuare i tempi dell'operazione di \textsc{Find-Set}
(che viene utilizzata anche in \textsc{Union} per individuare il rappresentante sul quale
cambiare padre), andando a ridurre i cammini dai singoli nodi verso la radice dell'albero.
Ciò viene fatto nell'operazione di \textsc{Find-Set} che diventa una funzione ricorsiva: man
mano che si risale l'albero i nodi attraversati vengono salvati e una volta raggiunta la radice
i padri dei nodi salvati vengono aggiornati con la radice individuata.\newline

Con la compressione dei cammini, la struttura dati ha un costo computazionale pari a:
\begin{equation}
    \Theta(n + f \cdot (1 + \log_{2+\frac{f}{n}}n) \footnote{\fullcite{forestPathCompression}} \label{FPCCost}
\end{equation}

con $f$ il numero di operazioni \textsc{Find-Set} e $n$ il numero di operazioni \textsc{Union}.

Nel caso della ricerca di componenti connesse $n$ è dato dal numero delle iterazioni del
ciclo \textit{for} alla riga \ref{alg:firstfor}, mentre $f$ dipende dal numero di iterazioni
del ciclo \textit{for} alla riga \ref{alg:secondfor} abbiamo:

\begin{equation}
    \label{FPCValues}
    \begin{aligned}
        n & = |V| = \Theta(|V|) = \Theta(v)   \\
        f & = 2 |E| = \Theta(|E|) = \Theta(e)
    \end{aligned}
\end{equation}

Quindi, per \eqref{FPCCost}, il costo totale è dato da:

\begin{equation}
    \label{FPC2Cost}
    \begin{aligned}
        \Theta(n+f\cdot (1+\log_{2+\frac{f}{n}}n)) & = \Theta(v + e (1+\log_{2+\frac{e}{v}}v)) \\
                                                   & = \Theta(v + e + 2e\log_{2+\frac{e}{v}}v) \\
                                                   & = O(v + e\log_{2+\frac{e}{v}}v)
    \end{aligned}
\end{equation}

Considerando il costo rispetto al solo numero dei vertici usando la copertura lineare, per \eqref{linearCoverage} si ha:
\begin{equation}
    \label{linearFPC}
    \begin{aligned}
        O(v + e\log_{2+\frac{e}{v}}v) & = O(v + v\log_{2+\frac{v}{v}}v) \\
                                      & = O(v + v\log_3v)               \\
                                      & = O(v\log_3v)
    \end{aligned}
\end{equation}

mentre usando la copertura quadratica, per \eqref{quadraticCoverage} si ha:

\begin{equation}
    \label{quadraticFPC}
    \begin{aligned}
        O(v + e\log_{2+\frac{e}{v}}v) & = O(v + v^2\log_{2+\frac{v^2}{v}}v) \\
                                      & = O(v + v^2\log_{2+v}v)             \\
                                      & = O(v^2\log_{2+v}v)
    \end{aligned}
\end{equation}
